% ---- Präambel mit Angaben zum Dokument
\documentclass[
	fontsize=12pt,           % Leitlinien sprechen von Schriftgröße 12.
	paper=A4,
	twoside=false,
	listof=totoc,            % Tabellen- und Abbildungsverzeichnis ins Inhaltsverzeichnis
	bibliography=totoc,      % Literaturverzeichnis ins Inhaltsverzeichnis aufnehmen
	titlepage,               % Titlepage-Umgebung anstatt \maketitle
	headsepline,             % horizontale Linie unter Kolumnentitel
	abstract,              % Überschrift einschalten, Abstract muss in {abstract}-Umgebung stehen
]{scrreprt}                  % Verwendung von KOMA-Report
\usepackage[utf8]{inputenc}  % UTF8 Encoding einschalten
\usepackage[ngerman]{babel}  % Neue deutsche Rechtschreibung
\usepackage[T1]{fontenc}     % Ausgabe von westeuropäischen Zeichen (auch Umlaute)
\usepackage{microtype}       % Trennung von Wörtern wird besser umgesetzt
\usepackage{lmodern}         % Nicht-gerasterte Schriftarten (bei MikTeX erforderlich)
\usepackage{graphicx}        % Einbinden von Grafiken erlauben
\usepackage{wrapfig}         % Grafiken fließend im Text
\usepackage{setspace}        % Zeilenabstand \singlespacing, \onehalfspaceing, \doublespacing
\usepackage[
	%showframe,                % Ränder anzeigen lassen
	left=2.7cm, right=2.5cm,
	top=2.5cm,  bottom=2.5cm,
	includeheadfoot
]{geometry}                      % Seitenlayout einstellen
\usepackage{scrlayer-scrpage}    % Gestaltung von Fuß- und Kopfzeilen
\usepackage{acronym}             % Abkürzungen, Abkürzungsverzeichnis
\usepackage{titletoc}            % Anpassungen am Inhaltsverzeichnis
\contentsmargin{0.75cm}          % Abstand im Inhaltsverzeichnis zw. Punkt und Seitenzahl
\usepackage[                     % Klickbare Links (enth. auch "nameref", "url" Package)
  hidelinks,                     % Blende die "URL Boxen" aus.
  breaklinks=true                % Breche zu lange URLs am Zeilenende um
]{hyperref}
\usepackage[hypcap=true]{caption}% Anker Anpassung für Referenzen
\urlstyle{same}                  % Aktuelle Schrift auch für URLs
% Anpassung von autoref für Gleichungen (ergänzt runde Klammern) und Algorithm.
% Anstatt "Listing" kann auch z.B. "Code-Ausschnitt" verwendet werden. Dies sollte
% jedoch synchron gehalten werden mit \lstlistingname (siehe weiter unten).
\addto\extrasngerman{%
	\def\equationautorefname~#1\null{Gleichung~(#1)\null}
	\def\lstnumberautorefname{Zeile}
	\def\lstlistingautorefname{Listing}
	\def\algorithmautorefname{Algorithmus}
	% Damit einheitlich "Abschnitt 1.2[.3]" verwendet wird und nicht "Unterabschnitt 1.2.3"
	% \def\subsectionautorefname{Abschnitt}
}

% ---- Abstand verkleinern von der Überschrift 
\renewcommand*{\chapterheadstartvskip}{\vspace*{.5\baselineskip}}

% Hierdurch werden Schusterjungen und Hurenkinder vermieden, d.h. einzelne Wörter
% auf der nächsten Seite oder in einer einzigen Zeile.
% LaTeX kann diese dennoch erzeugen, falls das Layout ansonsten nicht umsetzbar ist.
% Diese Werte sind aber gute Startwerte.
\widowpenalty10000
\clubpenalty10000

% ---- Für das Quellenverzeichnis
\usepackage[
	backend = biber,                % Verweis auf biber
	language = auto,
	style = numeric,                % Nummerierung der Quellen mit Zahlen
	sorting = none,                 % none = Sortierung nach der Erscheinung im Dokument
	sortcites = true,               % Sortiert die Quellen innerhalb eines cite-Befehls
	block = space,                  % Extra Leerzeichen zwischen Blocks
	hyperref = true,                % Links sind klickbar auch in der Quelle
	%backref = true,                % Referenz, auf den Text an die zitierte Stelle
	bibencoding = auto,
	giveninits = true,              % Vornamen werden abgekürzt
	doi=false,                      % DOI nicht anzeigen
	isbn=false,                     % ISBN nicht anzeigen
    alldates=short                  % Datum immer als DD.MM.YYYY anzeigen
]{biblatex}
\addbibresource{Inhalt/literatur.bib}
\setcounter{biburlnumpenalty}{3000}     % Umbruchgrenze für Zahlen
\setcounter{biburlucpenalty}{6000}      % Umbruchgrenze für Großbuchstaben
\setcounter{biburllcpenalty}{9000}      % Umbruchgrenze für Kleinbuchstaben
\DeclareNameAlias{default}{family-given}  % Nachname vor dem Vornamen
\AtBeginBibliography{\renewcommand{\multinamedelim}{\addslash\space
}\renewcommand{\finalnamedelim}{\multinamedelim}}  % Schrägstrich zwischen den Autorennamen
\DefineBibliographyStrings{german}{
  urlseen = {Einsichtnahme:},                      % Ändern des Titels von "besucht am"
}
\usepackage[babel,german=quotes]{csquotes}         % Deutsche Anführungszeichen + Zitate


% ---- Für Mathevorlage
\usepackage{amsmath}    % Erweiterung vom Mathe-Satz
\usepackage{amssymb}    % Lädt amsfonts und weitere Symbole
\usepackage{MnSymbol}   % Für Symbole, die in amssymb nicht enthalten sind.


% ---- Für Quellcodevorlage
\usepackage{scrhack}                    % Hack zur Verw. von listings in KOMA-Script
\usepackage{listings}                   % Darstellung von Quellcode
\usepackage{xcolor}                     % Einfache Verwendung von Farben
\input{Inhalt/00_Latex/quellcodeStyle}  % Weitere Details sind ausgelagert

\usepackage{algorithm}                  % Für Algorithmen-Umgebung (ähnlich wie lstlistings Umgebung)
\usepackage{algpseudocode}              % Für Pseudocode. Füge "[noend]" hinzu, wenn du kein "endif",
                                        % etc. haben willst.

\makeatletter                           % Sorgt dafür, dass man @ in Namen verwenden kann.
                                        % Ansonsten gibt es in der nächsten Zeile einen Compilefehler.
\renewcommand{\ALG@name}{Algorithmus}   % Umbenennen von "Algorithm" im Header der Listings.
\makeatother                            % Zeichen wieder zurücksetzen
\renewcommand{\lstlistingname}{Listing} % Erlaubt das Umbenennen von "Listing" in anderen Titel.

% ---- Tabellen
\usepackage{booktabs}  % Für schönere Tabellen. Enthält neue Befehle wie \midrule
\usepackage{multirow}  % Mehrzeilige Tabellen
\usepackage{siunitx}   % Für SI Einheiten und das Ausrichten Nachkommastellen
\sisetup{locale=DE, range-phrase={~bis~}, output-decimal-marker={,}} % Damit ein Komma und kein Punkt verwendet wird.
\usepackage{xfrac} % Für siunitx Option "fraction-function=\sfrac"

% ---- Für Definitionsboxen in der Einleitung
\usepackage{amsthm}                     % Liefert die Grundlagen für Theoreme
\usepackage[framemethod=tikz]{mdframed} % Boxen für die Umrandung
\input{Inhalt/00_Latex/highlightBoxen}  % Weitere Details sind ausgelagert

% ---- Für Todo Notes
\usepackage{todonotes}
\setlength {\marginparwidth }{2cm}      % Abstand für Todo Notizen


% ---- Elektronische Version oder Gedruckte Version?
% ---- Unterschied: Die elektronische Version enthält keinen Platzhalter für die Unterschrift
\usepackage{ifthen}
\newboolean{e-Abgabe}
\setboolean{e-Abgabe}{false}    % false=gedruckte Fassung

% ---- Persönlichen Daten:
\newcommand{\titel}{Chess of Duty}
\newcommand{\titelheader}{Programmentwurf - Chess of Duty}
\newcommand{\arbeit}{Advanced Software-Engineering}
\newcommand{\studiengang}{Informatik}
\newcommand{\studienjahr}{2023}
\newcommand{\autor}{Clemens Richter \& Johannes Peters}
\newcommand{\autorReverse}{Richter, Clemens \& Peters, Johannes}
\newcommand{\verfassungsort}{Karlsruhe}
\newcommand{\matrikelnr}{7661745 \& 5802185}
\newcommand{\kurs}{TINF20B1}
\newcommand{\bearbeitungsmonat}{April 2023}
\newcommand{\abgabe}{30. April 2023}
\newcommand{\bearbeitungszeitraum}{04.10.2022 - 30.04.2023}
\newcommand{\betreuerDhbw}{Herr Daniel Lindner}

\input{Inhalt/00_Latex/kopfundFusszeile}

% ---- Hilfreiches
\newcommand{\zB}{z.\,B. }   % "z.B." mit kleinem Leeraum dazwischen (ohne wäre nicht korrekt)
\newcommand{\dash}{d.\,h. }

\newcommand{\code}[1]{\texttt{#1}} % Ist einfacher zu schreiben als ständig \texttt und erlaubt
                                   % Änderungen im Nachhinein, wenn man z.B. Inline-Code anders stylen möchte.

% ---- Silbentrennung (falls LaTeX defaults falsch / nicht gewünscht sind)
\hyphenation{Graph-Script} % anstatt GraphS-cript

% ---- Beginn des Dokuments
\begin{document}
\setlength{\parindent}{0pt}              % Keine Paragraphen Einrückung.
                                         % Dafür haben wir den Abstand zwischen den Paragraphen.
\setcounter{secnumdepth}{2}              % Nummerierungstiefe fürs Inhaltsverzeichnis
\setcounter{tocdepth}{1}                 % Tiefe des Inhaltsverzeichnisses. Ggf. so anpassen,
                                         % dass das Verzeichnis auf eine Seite passt.
\sffamily                                % Serifenlose Schrift verwenden.

% ---- Vorspann
% ------ Titelseite
\singlespacing
\thispagestyle{empty}
\begin{titlepage}
\enlargethispage{4cm}

\begin{figure}
	\centering
	\includegraphics[height=5cm]{Bilder/Logos/Logo_DHBW.pdf} 
\end{figure}
		
\vspace*{0.1cm}

\begin{center}
	\huge{\textbf{\titel}}\\[1.5cm]
	\Large{\textbf{\arbeit}}\\[0.5cm]
	\normalsize{im Rahmen der Prüfung zum\\[1ex] \textbf{Bachelor of Science (B.Sc.)}}\\[0.5cm]
	\Large{des Studienganges \studiengang}\\[1ex]
	\normalsize{an der Dualen Hochschule Baden-Württemberg Karlsruhe}\\[1cm]
	\normalsize{von}\\[1ex] \Large{\textbf{\autor}} \\[1cm]
\end{center}

\vspace*{0.4cm}

\begin{center}
	%\vfill
	\begin{tabular}{ll}
		Abgabedatum:                     & \abgabe \\[0.2cm]
		Bearbeitungszeitraum:            & \bearbeitungszeitraum \\[0.2cm]
		Matrikelnummer, Kurs:            & \matrikelnr , \kurs \\[0.2cm]

		Betreuer der Arbeit:   & \betreuerDhbw \\[0.2cm]
	\end{tabular} 
\end{center}
\end{titlepage}
  % Titelseite
\newcounter{savepage}
\pagenumbering{Roman}                    % Römische Seitenzahlen
\onehalfspacing

% ------ Erklärung, Sperrvermerk, Abstact
%\include{Inhalt/01_Standard/erklaerung}
%\include{Inhalt/02_Abstract/abstract-de.tex}
%\include{Inhalt/02_Abstract/abstract-en.tex}

% ------ Inhaltsverzeichnis
\singlespacing
\tableofcontents

% ------ Verzeichnisse
\renewcommand*{\chapterpagestyle}{plain}
\pagestyle{plain}
\setcounter{savepage}{\value{page}}


% ---- Inhalt der Arbeit
\cleardoublepage
\pagenumbering{arabic}                  % Arabische Seitenzahlen für den Hauptteil
\setlength{\parskip}{0.5\baselineskip}  % Abstand zwischen Absätzen
\rmfamily
\renewcommand*{\chapterpagestyle}{scrheadings}
\pagestyle{scrheadings}
\onehalfspacing

%Hier kommen die \include-Statements
\chapter{Einleitung}

Im Rahmen des Moduls \glqq Advanced Software Engineering\grqq{} wurde ein Schachspiel als Projektgrundlage ausgewählt. 
Chess of Duty ist ein Offline-Multiplayer-Schachspiel für zwei Personen. 
Das Hauptziel des Programmentwurfs besteht darin, Schach gemäß den Standardregeln zu implementieren. 

Der Nutzen des Schachspiels für unsere Kunden entspricht dem von anderen Videospielen. 
Die Anwendung dient ausschließlich der Unterhaltung der Nutzer. 
Zusätzlich können die Anwender ihre strategischen Fähigkeiten und logisches Denken trainieren.

Das Schachspiel wird objektorientiert konzipiert und in Processing programmiert. 
Processing ist eine Open-Source-Programmiersprache, die auf Java basiert und einen besonderen Schwerpunkt auf die einfache Erstellung von Grafiken und Animationen setzt. 
Dadurch eignet sich Processing besonders für die Gestaltung interaktiver Benutzeroberflächen. 

\begin{balken}
    \tip
    \\
    Das GitHub-Repository ist mit folgendem Link erreichbar:\\
    \url{https://github.com/clemens1403/AdvSWE}
\end{balken}

\section{Vorwort}

Sehr geehrter Herr Lindner, anbei finden Sie unsere Ausarbeitung für den Programmentwurf aus dem fünften und sechsten Semester. 
Während der Projektarbeit sind verschiedene Herausforderungen aufgetreten, insbesondere bei der Projektwahl und der Auswahl des Technologiestacks. 
Für die Umsetzung einer Clean Architecture hätte sich im Nachhinein ein Verwaltungsprogramm als geeigneter erwiesen. 
Zudem gestaltete sich die Verwendung von Processing in einem Java-Projekt leider weniger intuitiv als von den Entwicklern angenommen. 
Insbesondere die Integration von Processing in IntelliJ bereitete längere Zeit Probleme.

Trotz dieser Schwierigkeiten ist das Ergebnis des Programmentwurfs ein Projekt, das die in der Vorlesung vermittelten Methoden und Prinzipien bestmöglich umsetzt. 
Sollte etwas nicht umgesetzt worden sein, wird darauf zumindest eingegangen. 

Die Implementierung der Schachlogik war äußerst umfangreich, weshalb einige Funktionen nicht unterstützt werden, beispielsweise das Schlagen en passant.
Die Durchführung eines einfachen Schachspiels ist jedoch problemlos möglich.

\section{Inbetriebnahme}

Für das Projekt, das auf GitHub einsehbar ist, wurde keine JAR-Datei erstellt. 
Das Projekt kann jedoch problemlos in IntelliJ oder Eclipse ausgeführt werden. 
Alle erforderlichen Abhängigkeiten sind in der pom-Datei des Maven-Projekts aufgeführt und im Repository verfügbar.

Es gibt eine wichtige Anmerkung, die beim Importieren der Abhängigkeiten und beim Ausführen des Projekts beachtet werden sollte. 
Abhängig von der verwendeten IDE und der Bildschirmgröße kann es vorkommen, dass Processing die Darstellung unterschiedlich skaliert, wodurch das Bild verzerrt erscheinen kann.
Dieses Problem tritt hauptsächlich bei kleinen Laptop-Bildschirmen auf, während größere Desktop-Monitore davon nicht betroffen sind.

Um das Problem zu beheben, muss in der Run-Konfiguration der Klasse \glqq Chess of Duty\grqq{} eine VM-Option angegeben werden. 
Diese Option kann ausgewählt werden, indem man auf \glqq Modify Options\grqq{} im Reiter \glqq Build and Run\grqq{} klickt und das entsprechende Feld auswählt.

\begin{minipage}{\linewidth}
    \centering
    \includegraphics[scale=0.45]{Bilder/erklaerung_01.PNG}
    \captionof{figure}{Eingabefeld für VM-Options}
\end{minipage}

Sobald in das Feld Eingaben getätigt werden können, muss folgender Parameter eingetragen und gespeichert werden, bevor die Konfigurationsübersicht geschlossen und das Programm normal ausgeführt werden kann.  
\begin{figure}[h!]
    \centering
    \texttt{-Dsun.java2D.uiScale=1.0}
\end{figure}

\begin{minipage}{\linewidth}
    \centering
    \includegraphics[scale=0.45]{Bilder/erklaerung_02.PNG}
    \captionof{figure}{VM-Options-Parameter zur richtigen Skalierung}
\end{minipage}



\chapter{Domain Driven Design}

\section{Ubiquitous Language}

Der Auftraggeber des Programmentwurfs ist ein deutscher Kunde. 
Obwohl in der Programmierung Englisch als die Standardsprache angenommen wird, wird aufgrund der Kundenlokalität die Projektsprache als \glqq Deutsch\grqq{} festgelegt. 
Dadurch wird versucht die meisten Ausdrücke aus der Domäne ins Deutsche zu übernehmen. 
Auch wenn dies für den ungeübten Programmierer, der das Programmieren nur in Englisch ausübt, eine zusätzliche Herausforderung darstellt und teilweise zu Namensfindungsschwierigkeiten oder längeren Funktionsnamen führen kann, wird an der Domänensprache festgehalten, um dem Kunden den Code so übersichtlich wie möglich übergeben zu können.

\section{Domainenbausteine}

\subsection*{Value Objects}

Ein Value Object ist ein unveränderliches Objekt, das einen bestimmten Wert repräsentiert und keine eigene Identität besitzt. 
Es definiert sich durch seine Eigenschaften und nicht durch eine eindeutige Identität, wodurch es austauschbar und vergleichbar wird.

Im Standardschach repräsentiert ein \textbf{Feld} eine Position, die durch eine Kombination aus \texttt{Spalte} und \texttt{Zeile} und \texttt{Farbe} definiert ist. 
Da Schachfelder keinen Lebenszyklus haben und ihre Gleichheit durch ihre Position bestimmt wird, werden sie als Value Objects geführt. 
Weiterhin ist es besonders wichtig für die grundlegende Funktion eines Schachspiels, dass Schachfelder als unveränderliche Objekte behandelt werden, da ein Schachspiel auf einer \glqq festen Unterlage\grqq{} spielt wird. 

Ein \textbf{Schachbrett} kann ebenfalls als Value Object betrachtet werden, weil es einen unveränderlichen Zustand repräsentiert. 
Ein Schachbrett ist in der domainenspezifischen Abbildung eines Schachspiels von der Realität in Programmcode eine übergeordnete Verwaltungsstruktur der einzelnen Felder.
Dabei werden 64 Schachfelder definiert und als zweidimensionale Spielebene dargestellt.
Die Gleichheit von zwei Schachbrettern basiert auf der Gleichheit ihrer Felder, unabhängig der tatsächlichen Instanz. 
Da die Felder jedoch als Value Object geführt werden, kann dies auch auf das Schachbrett übertragen werden. 
Ds Schachbrett weist keine Verhaltensänderungen auf, sodass ide Integrität des Objekts die ganze Zeit über gewahrt werden muss. 

Ein \textbf{Schachzug} setzt sich aus einer ausgewählten \texttt{Figur}, der \texttt{Startposition}, der \texttt{Endposition} und weiteren Informationen über den Einfluss der Bewegung der betreffenden Figur auf umliegende Figuren zusammen. 
Dazu zählen Informationen darüber, ob eine andere Figur geschlagen wird, ob durch den Zug Schach geboten wird, ob eine Bauernumwandlung stattfindet oder ob andere Figuren ebenfalls bewegt werden müssen, wie es bei der Rochade der Fall ist. 
Die Implementierung eines Schachzugs dient der Protokollierung einer gespielten Schachpartie. 
Ein Schachzug wurde als Value Object klassifiziert, da er keinen erkennbaren Lebenszyklus besitzt und seine Eigenschaften nach der Erstellung nicht mehr verändert werden.

Der \textbf{Spielzug} ein Value Object, welches jeweils zwei Schachzüge mit einer ID zusammenfasst. 

Der Spielzug lässt sich eindeutig durch einen \texttt{zugNummer} identifizieren, die mit jedem Spielzug so lange um eins hochzählt wie das Schachspiel geht. 
Auch wenn somit jeder Spielzug einen eindeutigen Schlüssel aufweist, der mit jedem erzeugten Spielzug um einen Wert inkrementiert, wird der Spielzug als Value Object geführt. 
Grund dafür ist die immutable Art des Spielzugs, welcher nach Erstellung genau in der Form, in der er erstellt wurde, in die Protokollierung des Spiels geschrieben wird. 

\newpage

\subsection*{Entities}

Eine Entity ist ein Objekt, das innerhalb der Domaine eine einzigartige Identität besitzt und sich im Laufe der Zeit verändern kann. 
Entities werden durch ihre Identität und nicht durch ihre Attribute definiert und weisen einen Lebenszyklus auf.

Im Domainencode gibt es die Klasse \textbf{Figur}. 
Diese dient als abstrakte Oberklasse für alle Figuren in einem Schachspiel. 
Da die Klasse Figur als \textit{abstract class} implementiert ist, kann sie nicht instanziiert werden.
Daher sind alle Spielfiguren, die von Figur erben, als Entitäten zu betrachten. 
Zu diesen Spielfiguren gehören \textbf{König, Dame, Turm, Läufer, Springer} und \textbf{Bauer}. 
Jede Figur kann über das Attribut \texttt{position} eindeutig identifiziert werden. 
Die Position der Figuren fungiert als natürlicher Schlüssel. 
Auf einem Feld kann immer nur eine Figur stehen, wodurch die Eindeutigkeit gewährleistet ist. 
Wenn eine Figur auf ein Feld mit einer anderen Figur zieht, wird die dort stehende Figur geschlagen und aus dem aktiven Spiel entfernt. 
Auch in diesem Fall steht letztendlich nur eine Figur auf dem Feld. 
Innerhalb der Domaine \glqq Schach\grqq{} ist dieser Schlüssel veränderlich, da sich die Schachfiguren auf dem Spielfeld bewegen können. 
Die Verwendung eines Surrogatschlüssel wäre grundsätzlich auch möglich, aber in diesem Fall wird sich an der Domaine orientiert.
Im analogen Schach wird die Farbe einer Figur, der Figurentyp und die Position zur eindeutigen Identifizierung verwendet.   
Eine Schachfigur besitzt einen eigenen Lebenszyklus. Sobald ein Schachspiel startet, werden alle Figuren in ihrer Startposition instanziiert. Während des Spiels können sie nahezu unbegrenzt bewegt werden. Wenn eine Figur geschlagen wird, wird sie gelöscht.

Auch die \textbf{Spieler} können als Enität betrachtet werden. 
Ein Spieler kann durch seinen \texttt{namen} als natürlichen Schlüssel identifiziert werden. 
Während eines Schachspiels kann ein Spieler in jedem Zug eine Figur bewegen, weiterhin kann er aber auch das Spiel aufgeben. 
Weiterhin ist der Spieler für das Elosystem nötig und ist veränderbar in seinem Rang. 

%Hier muss noch was zum Schlüssel geschrieben werden
Ein \textbf{Schachspiel} sollte ebenfalls als Entität behandelt werden. 
Es besitzt eine eigene Identität und eine eigene Lebenszeit über eine Partie Schach hinweg, in dem es verschiedene Zustände annehmen kann. 
Das Schachspiel ist eine übergeordnete logische Ebene, welche Spieler, Schachbrett und Figuren zusammenlegt. 
Auch wenn beim Schach die Ausgangslage der Figuren immer gleich ist, so unterscheidet sich eine Partie Schach in ihrem Verlauf. 

\newpage

\subsection*{Aggregate}

\subsection*{Repositories}

\section{Domainenevents}
\chapter{Clean Architecture}

\section*{Schicht 4 - Abstraction Code}

\begin{itemize}
    \item Bewegungsmatrizen
\end{itemize}

\section*{Schicht 3 - Domain Code}

\begin{itemize}
    \item Dame
    \item König
    \item Turm
    \item Läufer
    \item Springer
    \item Bauer
    \item Schachbrett
    \item Feld
    \item ...
\end{itemize}

\section*{Schicht 2 - Application Code}

\section*{Schicht 1 - Adapters}

\section*{Schicht 0 - Plugins}

GUI


Mainklasse - Chess of Duty
\chapter{Entwurfsmuster}

Für \glqq Chess of Duty\grqq{} wurde ein Observer-Pattern eingebaut.

\begin{figure}[h!]
    Commit:  \scriptsize https://github.com/clemens1403/AdvSWE/commit/c04b243f87959c6f80f7d75ccf31cba30af50bd5
\end{figure}

Ein Observer, auf deutsch auch Beobachter genannt, ist ein Entwurfsmuster in der Software-Entwicklung die eine lose Kopplung zwischen Objekten ermöglicht.
Mit Observern können 1:N-Beziehungen hergestellt werden, wobei die Änderung an einem Objekt automatisch an alle abhängigen Objekte weitergegeben werden.
Das Pattern basiert auf zwei Bestandteilen:

\begin{itemize}
    \item Subject: Ist das Element im Code, welches als Beobachtungsgrundlage dient. 
    Das Subjekt enthält eine Liste aller registrierter Observer und stellt Methoden zur Registrierung, Entfernung und Benachrichtigung bereit. 
    Sobald das Subjekt Veränderungen erfährt, werden diese automatisch an alle registrierten Observer übergeben.
    \item Observer: Ein Observer implementiert die vom Subject definierte Schnittstelle, um Benachrichtigungen zu erhalten, um auf die Änderungen  der Subjects reagieren zu können.
\end{itemize}

Damit das Schachspiel nutzbar ist, müssen alle grafischen Elemente regelmäßig gezeichnet werden.
Dabei muss immer der neueste Zustand des Spiels dargestellt werden, welche die Informationen aus der Application-Schicht entnimmt und über die Plugin-Schicht zeichnet. 
Den Zeichner-Klassen müssen Änderungen des aktuellen Spielstandes mitgeteilt werden, sobald sie auftreten.

Im Folgenden wird sowohl der Stand vorher und der Stand nachher genauer beschrieben. 
Für die visuelle Darstellung werden UML-Diagramme verwendet. 
Im Kontext des gesamten Projekts sind die betrachteten Klassen ziemlich umfangreich und weisen eine hohe Anzahl an Attributen und Funktionen.
Um die Diagramme nicht mit unnötiger Komplexität zu überladen, wurde die Darstellung auf die wichtigsten Klassenbestandteile für die Implementierung des Observer-Pattern reduziert. 

\newpage

\section{Vor dem Entwurfsmuster}

Auch ohne die Implementierung des beschriebenen Beobachter-Musters musste für den funktionierenden Programmablauf der aktuellste Spielzustand aus der Spiellogikklasse an die Zeichnerklassen übergeben werden, um die grafische Oberfläche korrekt zu zeigen. 
Processing besitzt eine draw()-Methode, die zur Zeichnung von Grafiken genutzt werden kann. 
In dieser Methode werden die einzelnen Zeichnerklassen koordiniert, sodass alle grafischen Elemente mehrmals die Sekunde gezeichnet werden. 

\begin{minipage}{\linewidth}
    \centering
    \includegraphics[scale=0.75, trim={0 25cm 0 0}]{Bilder/SWE_ohne_Observer.pdf}
    \captionof{figure}{Programmausschnitt ohne Observer-Pattern}
\end{minipage}

Nach ursprünglicher Implementierung wurde bei jedem Aufruf der draw()-Methode in der Zeichenklasse die Funktion \texttt{setSchachspielKontrollierer} aufgerufen.
Die Methode der Klasse \texttt{SchachspielZeichner} nimmt \texttt{SchachspielKontrollierer} entgegen und setzt diesen als globale Variable in der Zeichnerklasse. 
Da jede übergebene Kontrollierer-Instanz die Variable \texttt{Schachspiel} enthält, kann auch diese Variable lokal in der Zeichnerklasse gesetzt werden, damit auf deren Basis die Grafiken erstellt werden.

\newpage

\section{Mit dem Entwurfsmuster}

Um ein Beobachter-Pattern einzubauen, werden zwei zusätzliche Klassen benötigt, ein Interface für den Beobachter und ein Interface für das Subjekt.
Für den Programmentwurf wurde ein Beobachtermuster für das Attribut Schachspiel erstellt, um alle Änderungen zu registrieren.

Eine Interfaceklasse für einen \texttt{Beobachter} beschreibt die Methode \texttt{aktualisiere()}.
Das Interface für das Subjekt umfasst Methoden zum Registrieren, zum Entkoppeln und zur Benachrichtigung von Beobachtern. 
Diese Methoden müssen bei der Verwendung der Interfaces in verschiedenen Klassen überschrieben werden, um deren Funktion genauer zu beschreiben. 

\begin{minipage}{\linewidth}
    \centering
    \includegraphics[scale=0.75, trim={0 20cm 0 0}]{Bilder/SWE_mit_Observer.pdf}
    \captionof{figure}{Programmausschnitt mit Observer-Pattern}
\end{minipage}

In der eigenen Implementierung für das Schachprogramm verwendet die Zeichnerklasse \texttt{SchachspielZeichner} das Interface des Beobachters.
Die Spiellogik-steuernde Klasse \texttt{SchachspielKontrollierer} implementiert das Interface für das Subjekt. 
Da beide Klassen jeweils ein Interface verwenden, müssen die in den Interfaces beschriebenen Methoden überschrieben und ausprogrammiert werden. 

Im Kontrollierer wird eine Liste an Beobachtern angelegt.
Die bereits im Interface beschriebenen Methoden \texttt{registriereBeobachter()} und \texttt{unregistriereBeobachter()} verwalten die Beobachter-Instanzen, sofern ein neuer Beobachter hinzugefügt oder ein bestehender Beobachter entfernt werden soll. 
\texttt{benachrichtigeBeobachter()} ist eine Methdoe, bei der durch die Liste an Beobachtern iteriert und für jeden Beobachter die Funktion \texttt{aktualisiereSchachspiel()} aufgerufen wird.

In der Zeichnerklasse wird das Beobachter-Interface verwendet.
Dabei muss die Methode \texttt{aktualisiereSchachspiel()} überschrieben werden. 
Dabei übernimmt diese neue Methode die ursprüngliche Funktion der Methode \texttt{setSchachspielKontrollierer}.
Für die genaue Umsetzung bedeutet es, dass diese neu geschriebene Methode durch den Beobachteraufruf einen neuen Kontrollierer übertragen bekommt, aus welchem die neueste Version von \texttt{schachspiel} entnommen werden kann, nachdem eine Änderung an diesem Attribut auftritt. 

Wenn man die Stände von vorher und nachher betrachtet, könnte man meinen, dass kein wirklicher Mehrwert geschaffen wurde, sondern nur eine höhere Komplexität im Code eingebaut wurde. 
Doch dem ist nicht so. 
Anfänglich wurde bei jedem Zeichenaufruf durch die draw()-Methode von Processing das Attribut \texttt{schachspiel} an die Zeichnerklasse übergeben. 
Das Observer-Pattern verhindert nun, dass mehrmals die Sekunde unnötigerweise das Attribut neu gesetzt wird. 
Mit den eingebauten Beobachter für das Schachspiel-Attribut wird die \texttt{schachspiel} nur dann neu gesetzt, wenn sie explizit vom Subjekt übergeben wird.
Eine Übergabe wird dabei nur dann ausgelöst, wenn sich das Schachspiel-Attribut ändert. 

\chapter{Unit Tests}
Dieses Kapitel befasst sich mit den Unit-Tests des Programmes. Dazu werden zuerst die ATRIP-Regeln erläutert und deren Anwendung beschrieben. Im Anschluss wird die Code Coverage erläutert. Zum Schluss wird noch auf die Verwendung von Mock-Objekten in den Unit-Tests eingegangen.

\section{ATRIP-Regeln}
Die ATRIP-Regeln sind Regeln, welche gute Unit-Tests einhalten sollten.
Die Regeln setzen sich wiefolgt zusammen:
\begin{itemize}
	\item Automatic - Eigenständig
	\item Thorough - Gründlich (genug)
	\item Repeatable - Wiederholbar
	\item Independent - Unabhängig voneinander
	\item Professional - Mit Sorgfalt hergestellt
\end{itemize}

\subsection{Automatic}
Die Regel \glqq{}Automatic\grqq{} besagt, dass die Tests ohne manuelle Eingriffe selbstständig ablaufen müssen. Weiterhin müssen die Tests auch ihre Ergebnisse selbst überprüfen. Als Ergebnisse sind dabei nur \glqq{}bestanden\grqq{} oder \glqq{}nicht bestanden\grqq{} zulässig. Durch die Anwendung dieser Regel ist es möglich Unit Tests zu automatisieren. \pagebreak

\lstinputlisting[
	label={code:automatic},
	caption={Automatic Unit Test},
	captionpos=b,
	style=EigenerJavaStyle
]{Quellcode/automatic.java}

In Quellcode \ref{code:automatic} ist ein Unit Test aus der Klasse BauerDienstTest zu sehen. In den Zeilen 14-16 überprüft der Test mithilfe von Assertions selbst sein Ergebnis. Das eigenstädnige ablaufen des Tests wird über Maven gewährleistet. Somit ist der Test Automatic.

\subsection{Thorough}
Die Regel \glqq{}Thorough\grqq{} besagt, dass alles Notwendige getestet werden muss. Die Definition von notwendig hängt dabei immer mit den Rahmenbedingungen zusammen. Mindestens muss aber jede missionskritische Funktionalität getestet werden und für jeden aufgetretenen Fehler muss ein Testfall existieren, welcher das erneute Auftreten des Fehlers verhindert. Durch diese Regel entstehen zusätzliche Tests im Umfeld von einem Fehler, um weitere Fehler zu verhindern.

Für den konkreten Fall des Schachspiels ist es notwendig jegliche Figuren-Dienste sowie den Kontrollierer für das Schachspiel zu testen. Daher sind für alle diese Klassen Unit Tests erstellt worden. Da es sich bei dem Programm um ein Spiel handelt, ist nahezu jede Funktion die Klassen missionskritisch. Da der Aufwand für jede Methode einen Unit Test zu programmieren zu hoch für dieses Projekt wäre, wurden zu jeder Klasse der Figuren-Dienste ein bis zwei Unit Tests implementiert.

\subsection{Repeatable}
Die Regel \glqq{}Repeatable\grqq{} besagt, dass jeder Test automatisch durchführbar sein sollte und dabei stets das gleiche Ergebnis liefern sollte. Dazu muss der Test unabhängig von der Umgebung sein. Dabei sind vor allem der Umgang mit einem Datum oder mit Zufallszahlen problematisch. Auch der Zugriff auf das Dateisystem stellt eine Abhängigkeit von der Umgebung dar.

Bei den implementierten Unit Tests gibt es keine Abhängigkeiten von der Umgebung. Die standardmäßigen Problemquellen sind für das Testen der Anwendung irrelevant, da weder Daten oder Zufallszahlen für das Testen des Schachspiels relevant sind noch ein Zugriff auf das Dateisystem in den Tests erfolgt. Das Spiel ist in sich selbst abgeschlossen, weshalb die Umgebung automatisch immer gleich ist.

\subsection{Independent}
Die Regel \glqq{}Repeatable\grqq{} besagt, dass Tests unabhängig von einander funktionieren müssen. Reihenfolge und Zusammenstellung müssen für ds ausführen der Tests irrelevant sein. Im Idealfall testet jeder Test genau einen Aspekt der zu testenden Komponente.

Da jeder Test alle benötigten Abhängigkeiten für sich selbst erzeugt erzeugt (mit zum Beispiel Mocks) gibt es keine Abhängigkeiten zu anderen Tests.

\subsection{Professional}
Die Regel \glqq{}Repeatable\grqq{} besagt, dass Testcode zum relevanten Produktionscode gehört und so leicht verständlich wie mögich sein sollte.

Um dies umzusetzen wurden zum einen sinnvolle Namen für die Testmethoden vergeben. So sind die Namen der Tests stets so aufgebaut, dass sie aus Methodenname + \glqq{}Test\grqq{} bestehen. Weiterhin sind die Tests an sich alle stets gleich aufgebaut. Zuerst werden die benötigten Abhängigkeiten erzeugt. Im Anschluss wird die zu testende Funktion ausgeführt und zum Schluss wird das Ergebnis überprüft. Durch diese Einheitliche Struktur wird für die Lesbarkeit der Tests gesorgt.  Weiterhin trägt auch eine simple Benennung von Variablen für gute Lesbarkeit.

\section{Code Coverage}
Die Code Coverage gibt an wie viel Quellcode mit Tests abgedeckt ist. Um die Code Coverage zu ermitteln kann eine Code Coverage Analyse in einer IDE ausgeführt werden. Bei Code Coverage unterschiedet man zwischen Line Coverage und Branch Coverage. Bei Line Coverage wird die Anzahl der getesten Zeilen des Quellcodes ins Verhältnis zur Gesamtzahl der Zeilen des Quellcodes gestellt. Branch Coverage hingegen beschreibt die Abdeckung von Verzweigungen im Code (if-statements). Wenn ein Test nur einen von zwei Pfaden abdeckt so liegt die Branch Coverage bei nur 50 Prozent. Bei der Code Coverage ist es wichtig dass stets angegeben wird, ob Line Coverage oder Branch Coverage verwendet wurde, da sich das Ergebnis dieser beiden Verfahren stark unterscheiden kann.

In Abbildung \ref{fig:CodeCoverage} sind die Ergebnisse einer Code Coverage Analyse für die Figuren-Dienste zu sehen. Dabei werden alle Klassen abgedeckt. Auch die Abdeckung der Methoden ist mit 94 Prozent sehr gut. Bei der Line Coverage kommen die Tests auf 59 Prozent und bei Branch Coverage nur auf 37 Prozent.
\begin{figure}[ht]
	\centering
	\includegraphics[width=0.8\textwidth]{Bilder/CodeCoverage.png} 
	\caption{Code Coverage-Analyse in IntelliJ für die Figuren-Dienste}
	\label{fig:CodeCoverage}
\end{figure}


\section{Mocks}
Als Mock-Objekte werden Stellvertreter für echte Objekte bezeichnet. Mithilfe dieser Stellvertreter können Abhängigkeiten bei der Durchführung von Tests ersetzt werden. Durch das ersetzen der Abhängigkeiten einer Klasse mit Mocks wird das isolierte Testen dieser Klasse möglich. Da es sehr aufwendig ist Mocks selber zu programmieren werden häufig Mock-Tools verwendet. Für den Einsatz eines Mocks muss das Mock-Objekt zuvor trainiert werden. Insgesamt durchläuft ein Mock-Objekt somit drei Phasen: Training-Phase, Einsatz-Phase und Verifikation-Phase. 

Für das Schachspiel wurde Mockito als Mock-Tool verwendet. Ein Beispiel für die Verwendung von Mocks ist in der Testklasse KoenigDienstTest zu sehen (siehe Quellcode \ref{code:Mocks} oder Github). 

\lstinputlisting[
	label={code:Mocks},
	caption={Verwendung von Mocks beim Testen},
	captionpos=b,
	style=EigenerJavaStyle
]{Quellcode/mocks.java}

In Quellcode \ref{code:Mocks} sind zwei Methoden aus der Klasse KoenigDienstTest zu sehen. Die Methode setUp (Zeile 1-8) wird vor jeder Testfunktion ausgeführt und initialisiert alle für die Tests benötigten Objekte. Unter diesen Objekten befindet sich neben einem KoenigDienst-Objekt und und einem Schachbrett-Objekt auch ein Mock-Objekt der Klasse Koenig. Das Mock-Objekt wird in Zeile 5 von Quellcode \ref{code:Mocks} erzeugt und in den Zeilen 6 und 7 eingelernt. In der Test-Methode getMoeglicheZuegeTest (Zeile 10-25 in Quellcode \ref{code:Mocks}) werden zunächst Testspezifische Mock-Objekte erzeugt. Da mit dieser Methode Interaktionen mit anderen Schachfiguren überprüft werden sollen, wird jeweils eine weiße und eine schwarze Dame als Mock-Objekt erzeugt (Zeile 12-13). In den Zeilen 14 bis 17 werden diese beiden Mock-Objekte eingelernt. Abgespielt werden die Mocks mit dem Funktionsaufruf \emph{koenigDienst.getMoeglicheZuege(figuren, schachbrett, koenigMock)} in Zeile 22. Im Anschluss wird überprüft, ob die angelernten Aufrufe mindestens ein mal genutzt wurden (Zeile 24 bis 31).
\chapter{Refactoring}
Dieses Kapitel befasst sich mit dem Refactoring der Anwendung. Dazu werden zuerst Code Smells identifiziert und erläutert, warum diese so schlecht sind. Danach wird darauf eingegangen wie diese Code Smells durch Refactoring behoben werden können und wie die Probleme des Code Smells dadurch aufgehoben werden.

\section{Code Smells}
Bie Code Smells handelt es sich um eine verbesserungswürdige Quellcodestelle. Um festzustellen, ob eine Quellcodestelle verbesserungswürdig ist, gibt es keine festen Messwerte. Stattdessen werden solche Stellen hauptsächlich durch Erfahrung, aber auch teilweise durch Algorithmen oder Heuristiken lokalisiert. Code Smells können in verschiedene Kategorien gegliedert werden: Duplicated Code, Long Method, Large Class, Shotgun Surgery, Switch Statement und Code Comments. In den folgende Abschnitten werden Duplicated Code, Long Method und Switch Statement genauer an einem Beispiel aus dem Quellcode der ChessOfDuty-Anwendung erläutert.

\subsection{Code Duplication}
Eine Code Duplication beschreibt eine Stelle im Quellcode, an welcher doppelter Code vorhanden ist. Dies ist sehr problematisch, da im Falle einer Änderung alle Stellen des duplizierten Codes geändert werden müssen. Somit entsteht ein vielfacher Pflegeaufwand. Ein extremer Fall von Code Duplication ist in Quellcode \ref{code:codeduplication} zu sehen.  
\lstinputlisting[
	label={code:codeduplication},
	caption={Beispiel für Code Duplication},
	captionpos=b,
	style=EigenerJavaStyle
]{Quellcode/codeduplication.java}

Der Quellcode \ref{code:codeduplication} entstammt aus der SpringerDients-Klasse. Der Quellcode zeigt die Methode getMoeglicheZuege, welche die Züge berechnet, welche der entsprechende Springer machen kann. Bei der initialen implementierung wurde für jede grundsätzliche Schrittmöglichkeit eine eigene Methode implementiert, welche ausgibt bestimmt ob dieser Schritt möglich ist. Diese Methoden sind nahezu identisch. Zwei dieser Methoden sind mit zweiNachVorneEinsNachLinks und zweiNachVorneEinsNachRechts ebenfalls in Quellcode \ref{code:codeduplication} zu sehen. Wenn man zum Beispiel das Bewegungsmuster des Springers ändern wöllte müsste man in jeder dieser 8 Methoden etwas ändern. 


\subsection{Long Method}
Eine Long Method ist eine lange Methode. Lange Methoden sind schlecht, weil sie unübersichtlich sind und somit die Verständlichkeit des Quellcodes sinkt. Dadurch wird die Wartbarkeit des Codes schlechter und die Entwicklungsgeschwindigkeit sinkt. Ein Beispiel für eine Long Method ist in der Klasse BauerDienst mit der Methode getMoeglicheZuege zu finden (siehe Quellcode \ref{code:longmethod}). Die Methode erstreckt sich über unglaubliche 87 Zeilen, wodurch es sehr schwierig ist nachzuvollziehen, was genau in der Methode passiert.

\lstinputlisting[
	label={code:longmethod},
	caption={Beispiel für Long Method},
	captionpos=b,
	style=EigenerJavaStyle
]{Quellcode/longmethod.java}

\subsection{Switch-Statement}
Switch-Statements haben verschiedene Probleme.
Das erste Problem ist, dass meist gleiche Switch-Statements an mehreren Stellen verwendet werden. Weiterhin können Switch-Statements nicht aufgeteilt werden, sondern nur wachsen, wodurch eine hohe Komplexität entsteht. Zuletzt ist auch die Syntax sehr Fehleranfällig durch die Verwendung von Breaks oder Fall-Throughs. Ein Beispiel für die hohe Komplexität von Switch-Statements ist in Quellcode \ref{code:switchstatement} aus der Klasse ChessOfDuty zu sehen. Dargestellt ist die Methode mousePressed. In dieser Methode wird 

\lstinputlisting[
	label={code:switchstatement},
	caption={Beispiel für Switch-Statement},
	captionpos=b,
	style=EigenerJavaStyle
]{Quellcode/switchstatement.java}

\section{Refactoring der Code Smells}
In den folgenden Abschnitten wird erläutert, wie die Code Smells aus dem vorherigen Abschnitt behoben werden können. Dabei wird zum einen die Code Duplication aus Quellcode \ref{code:codeduplication} und zum anderen die Long Method aus Quellcode \ref{code:longmethod}


\subsection{Code Duplication}
Um eine Code Duplication zu beheben kann die Extract Method-Technik angewendet werden. Dabei werden zusammenhängende Quellcodefragmente in eigene Methoden ausgelagert Dadurch wird der Code Feingranularer und es bilden sich Abstraktionsstufen aus. Bei extrahierten Methoden sollte nach Möglichkeit auf Eingabe- und Ausgabe-Parameter verzichtet werden.

Im Quellcode \ref{code:codeduplication} ist es jedoch nicht sinnvoll eine Methode zu extrahieren, da die Methoden, in welchen die Code Duplication vorliegt, nahezu identisch sind. Daher werden die Methoden zu einer Methode, die alle 8 Methoden ersetzt, zusammengefasst. Dazu wird die Bewegung als zusätzliches Parameter übergeben und anstatt der fest definierten Integerwerte verwendet (siehe Quellcode \ref{code:fixduplication}). 

\lstinputlisting[
	label={code:fixduplication},
	caption={Code Duplication behoben},
	captionpos=b,
	style=EigenerJavaStyle
]{Quellcode/fixduplication.java}


\subsection{Long Method}
Um eine Long Method zu beheben kann ebebnfalls eine Extract Method angewendet werden. Alternativ können aber auch die Berechnungen von lokalen Variablen ausgelagert werden. Durch die Anwendung dieser Techniken konnte der Quellcode aus \ref{code:longmethod} von der Long Method befreit werden. Der neue Quellcode ist in Abbildung \ref{code:fixlongmethod} zu sehen.

\lstinputlisting[
	label={code:fixlongmethod},
	caption={Long Method behoben},
	captionpos=b,
	style=EigenerJavaStyle,
	firstline=1, 
	lastline=44
]{Quellcode/fixlongmethod.java}

Um die Methode getMoeglicheZuege zu verkürzen, wurde die Berechnung von Einzel- bzw. Doppelschritten des Bauern ausgelagert. Somit wird nur noch eine Methode aufgerufen, um zu bestimmen, ob der gewünschte Schritt möglich ist. Weiterhin wurde die Berechnung, ob der Bauer eine andere Figur schlagen kann, in eine neue Methode ausgelagert. Die neuen Methoden sind in Quellcode \ref{code:fixlongmethodextreaction} zu sehen. 


\lstinputlisting[
	label={code:fixlongmethodextreaction},
	caption={Long Method augelagerte Methoden},
	captionpos=b,
	style=EigenerJavaStyle,
	firstline=46, 
	lastline=90
]{Quellcode/fixlongmethod.java}

Mit der Methode berechneSchrittMoeglich wurde weiterhin eine Code Duplication behoben. Zuvor gab es bei der Berechnung der Möglichen Schritte eine Code Duplication (siehe Quellcode \ref{code:longmethod} Zeile 18-47 und Zeile 50-79).

Durch das Refactoring konnte die Long-Method um 43 Zeilen (nahezu 50 Prozent) verkürzt werden und der Code-Abschnitt im Allgemeinen konnte ebenfalls um 9 Zeilen durch die Entfernung der Code Duplication verkürzt werden. Somit konnte der verständlicher und wartbarer gemacht.

\chapter{Programming Principles}

\section{SOLID}
Die SOLID-Regeln wurden Anfang der Nullerjahre von Michael Feathers und Robert C. Martin gesammelt und formuliert. Die Regeln haben die Ziele Software wartbar, Systeme erweiterbar und Codebasen langlebiger zu machen.

\subsection{Single responsibility principle}
Das Single responsibility principle ist das Prinzip der einzigen Zuständigkeit. Somit sollte eine Klasse nur einen einzigen Grund haben sich zu ändern. Die Zuständigkeiten einer Klasse können mithilfe von Change dimensions dargestellt werden. Dabei spannt jede Zuständigkeit eine zusätzliche Achse auf. Entlang dieser Achsen werden Änderungen am Code dargestellt. Im Idealfall sind die Achsen daher orthogonal, da sich die Änderungen dadurch nicht gegenseitig beeinflussen. 

Dieses Prinzip wird im gesamten Quellcode angewendet. Die Klassen auf Domain-Ebene haben jeweils nur den Zweck das Entsprechende Objekt aus der Domain abzubilden. In der Application-Schicht hat jede Klasse Bezug zu einer Klasse aus der Domain-Ebene und nur die Aufgabe die entsprechenden Objekte zu Verwalten. Die einzige Ausnahme dabei ist der SchachspielKontrollierer, welcher die Koordination der anderen Klassen in der Application-Ebene übernimmt, um die Figuren auf dem Spielfeld miteinander interagieren zu lassen.

\subsection{Open/Closed principle}
Das Open/Closed principle besagt, dass Software-Entitäten offen für Erweiterungen sein sollen, aber gleichzeit geschlossen bezüglich Veränderungen sein sollen. Erweiterungen können zum Beispiel durch Unterklassen erschaffen werden, da nur die Unterklasse ihr Verhalten ändert, aber nicht die bereits existierende Klasse. Eine Veränderung stellt in diesem Kontext eine Modifikation des Codes durch geänderte Anforderungen dar. Zusammengefasst sollte bestehender Code also nicht mehr geändert werden müssen.

Als Beispiel für das Open/Closed principle im Code können die Figuren verwendet werden. Die Figuren sind so implementiert, dass beliebig neue Figuren hinzugefügt werden können oder Figuren ersetzt werden können.
\subsection{Liskov substitution principle}
Das LSP besagt, dass Objekte durch Instanzen ihrer Subtypen ersetzbar sein sollten, ohne die Korrektheit des Programms zu ändern. Somit gibt es strikte Regeln für Vererbungshierarchien. Subtypen dürfen dabei die Funktion der Oberklasse nicht einschränken sondern nur erweitern.

Als Beispiel für LSP können erneut die Figuren genommen werden. Zwischen der Figur-Klasse und den einzelnen Unterklassen tritt eine Kovarianz auf.


\subsection{Interface segregation principle}
Mithilfe von ISP sollen Schnittstellen mindestens in Nutzergruppen aufgeteilt werden. Dies wird umgesetzt indem anstelle eines großen Interfaces mehrere kleine Interfaces erstellt werden. Dies führt zu einer hohen Kohäsion und unterstützt auch das SRP. 

ISP wurde im Quellcode nicht angewendet, da keine Interfaces für das Spiel implementiert wurden. Die einzige sinnvolle Stelle für die verwendung eines Interfaces wäre bei den Diensten für die Figuren, da diese alle eine Methode getMoeglicheZuege implementieren. Allerdings würde dieses Interface nur eine einzelne Methode umfassen, weshalb die Anwendung von ISP hier nicht sinnvoll ist.

\subsection{Dependency inversion principle}
Das DIP besagt, dass Klassen höherer Ebenen nicht von Klassen niederer Ebe abhängig sein sollen, sondern beide im Idealfall von einem Interface abhängig sind. Dies verhindert, dass Änderungen aus einem niedrigerem Modul zu Änderungen in höheren Modulen führen. Umgesetzt wird das indem das hohe Modul eine Schnittstelle definiert, welche vom niedrigem Modul implementiert wird. Die Referenz auf die Konkrete Instanz wird dem höheren Modul dann per Dependency Injection übergeben. 

DIP wurde im Quellcode nicht angewendet. Die einzige Möglichkeit DIP anzuwenden besteht erneut bei den Figur-Diensten, welche für dieses Beispiel die niedrigeren Module darstellen. Das höhere Modul wäre der SchachspielKontrollierer. Der SchachspielKontrollierer könnte die verschiedenen Figur-Dienste als Interface definieren und die Figur-Dienste würden dann ein Interface mit der Methode getMoeglicheZuege definieren. Die Abhängigkeiten würden wie bisher per Dependency Injection übergeben werden. Jedoch wurde sich dazu entschieden dies nicht umzusetzen, da die Verständlichkeit des Codes dadurch sinkt (unserem empfinden nach). Durch die klare Typisierung der verschiedenen Figur-Dienste ist der Code strukturierter und man kann zusätzlich zum Name der Variable auch Anhand des Typs erkennen, welche Aufgabe der entsprechende Dienst hat.

\section{GRASP}
GRASP steht für General Responsibility Assignment Software Patterns/Principles und stellt Standardlösungen für Typische Fragestellungen bei der Softwareentwicklung dar. Insgesamt stellt GRASP neun Prinzipien/Werkzeuge bereit:
\begin{itemize}
    \item Low Coupling
    \item High Cohesion
    \item Indirection
    \item Polymorphism
    \item Pure Fabrication
    \item Controller
    \item Information Expert
    \item Creator
\end{itemize}

In den folgenden Abschnitten wird kurz das Konzept der Kopplung und das Konzept der Kohäsion erläutert und die Anwendung im Quellcode beschrieben.

\subsection{Kopplung}
Als Kopplung wird das Maß für die Abhängigkeit einer Klasse von ihrer Umgebung bezeichnet. Durch geringe Kopplung werden viele Vorteile ermöglicht. So lässt sich Code mit geringer Kopplung leicht anpassen und gut Testen. Weiterhin ist der Code leichter verständlich und kann besser wiederverwendet werden.

Der Quellcode ist eher ein negativ-Beispiel für geringe Kopplung. Dies kann daran erkannt werden, dass zum größten Teil statische Methodenaufrufe ausgeführt werden, keine Interfaces verwendet werden bis auf den Beobachter, und auch keine Events auf einem Eventbus versendet werden. Als größtes Negativ-Beispiel im Quellcode kann man den Schachspielkontrollierer nehmen, welcher an eine Vielzahl von anderen Klassen und Methoden gekoppelt ist.

\subsection{Kohäsion}
Bei Kohäsion handelt es sich um das Maß für den inneren Zusammenhalt einer Klasse. Dabei ist Kohäsion ein semantisches Maß - also abhängig von der menschlichen Einschätzung.

Die Kohäsion des Quellcodes ist relativ hoch. In den Klassen der Figur-Dienste rufen sich die Methoden zum größten Teil untereinander auf. Die Klassen der Figuren-Dienste können allgemein als positives Beispiel für eine hohe Kohäsion genommen werden.


\section{DRY}

DRY steht für Don't Repeat Yourself und besagt, dass man alles einmal machen soll und nur einmal machen soll. Das Motto von Dry ist kann wiefolgt definiert werden: \glqq{} Jeder Wissensaspekt darf nur eine einzige, nicht zweideutige verbindliche Repräsentation in einem System besitzen\grqq{}. Mechanische Duplikation ist dabei jedoch erlaubt, solange die Originalquelle klar definiert ist. Somit ändert eine Modifikation alle verknüpften Elemente in gleicher Weise, aber ändert keine nicht verknüpften Elemente.

Im Quellcode ist dieses Prinzip zum Beispiel in der Klasse SpringerDienst zu erkennen. Die Klasse definiert wiederverwendbare und universell anwendbare Methoden ohne sich zu wiederholen.

% ---- Literaturverzeichnis
\cleardoublepage
\renewcommand*{\chapterpagestyle}{plain}
\pagestyle{plain}
\pagenumbering{Roman}                   % Römische Seitenzahlen
\setcounter{page}{\numexpr\value{savepage}+1}

% ---- Anhang
\appendix
%\clearpage
%\pagenumbering{Roman}  % römische Seitenzahlen für Anhang

\newpage
\end{document}