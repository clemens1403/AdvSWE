\chapter{Einleitung}

Im Rahmen des Moduls \glqq Advanced Software Engineering\grqq{} wurde ein Schachspiel als Projektgrundlage ausgewählt. 
Chess of Duty ist ein Offline-Multiplayer-Schachspiel für zwei Personen. 
Das Hauptziel des Programmentwurfs besteht darin, Schach gemäß den Standardregeln zu implementieren. 
Dabei soll ein Leaderboard für die Spieler eingeführt werden, das auf einem Elo-System basiert.

Der Nutzen des Schachspiels für unsere Kunden entspricht dem von anderen Videospielen. 
Die Anwendung dient ausschließlich der Unterhaltung der Nutzer. 
Zusätzlich können die Anwender ihre strategischen Fähigkeiten und logisches Denken trainieren.

Das Schachspiel wird objektorientiert konzipiert und in Processing programmiert. 
Processing ist eine Open-Source-Programmiersprache, die auf Java basiert und einen besonderen Schwerpunkt auf die einfache Erstellung von Grafiken und Animationen setzt. 
Dadurch eignet sich Processing besonders für die Gestaltung interaktiver Benutzeroberflächen. 