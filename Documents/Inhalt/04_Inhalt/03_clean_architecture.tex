\chapter{Clean Architecture}

Das Projekt soll die Clean Architecture umsetzen, indem jede Schicht als eigenes Modul implementiert wird. 
Dadurch wird sichergestellt, dass nur von den äußeren Schichten auf die inneren zugegriffen werden kann und nicht umgekehrt. 

Das Projekt wird sich an der Schichtenarchitektur orientieren, wobei jedes Modul eine eigene Schicht repräsentiert. 
Der Einsatz von einzelnen Schichten ermöglicht eine fachliche Unabhängigkeit der Anwendung von der sonstigen Infrastruktur. 
Dadurch können die einzelnen Komponenten leichter wiederverwendet, getestet und weiterentwickelt werden. 
Zudem ist es einfacher, einzelne Komponenten der Infrastruktur auszutauschen. 
Dabei gilt: je weiter außenliegend, desto leichter austauschbar.

\section*{Schicht 4 - Abstraction Code}

Diese Schicht stellt den Kern der Applikation dar, wobei dies häufig durch die verwendete Programmiersprache realisiert wird, wie in Java beispielsweise durch Klassen wie String.
Im betrachteten Projekt konnte jedoch kein Abstraction Code identifiziert werden, die einigen mathematischen Konzepte eindeutig in den Domänen Code (Bewegungsregeln der Figuren) oder in den Adapter Code (Positionsberechnung für GUI-Elemente) einzuordnen sind. 

\section*{Schicht 3 - Domain Code}
%Rangliste fehlt hier noch eingeordnet

Der Domänencode beinhaltet alle bereits im Bereich \glqq Domain Driven Design\grqq{} besprochenen Entitäten, Value Objects, Aggregate und Repositories. 

Dazu gehören im Speziellen die folgenden Klassen:

\textbf{Value Objects:} Feld, Schachbrett, Schachzug

\textbf{Entitäten:} Bauer, Läufer, Springer, Turm, Dame, König, Spieler, Spielzug, Schachspiel

\textbf{Repositories:} SpielzugRepository, SpielerRepository

\textbf{Domain Services:} Bewegungsmatrizen, Bewegungsrichtung

\section*{Schicht 2 - Application Code}

\section*{Schicht 1 - Adapters}

\section*{Schicht 0 - Plugins}

GUI


Mainklasse - Chess of Duty