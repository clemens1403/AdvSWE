\chapter{Domain Driven Design}
\label{txt:ddd}

\section{Ubiquitous Language}

Die Ubiquitous Language bezeichnet das Vokabular, welches zwischen Domänenexperten
und Entwicklern, wodurch gewährleistet werden soll, dass die gleichen Begriffe in der
Domäne und im Sourcecode verwendet werden und Missverständnisse minimiert werden.

Der Auftraggeber des Programmentwurfs ist ein deutscher Kunde. 
Obwohl in der Programmierung Englisch als die Standardsprache angenommen wird, wird aufgrund der Kundenlokalität die Projektsprache als \glqq Deutsch\grqq{} festgelegt. 
Dadurch wird versucht, die meisten Ausdrücke aus der Domäne ins Deutsche zu übernehmen. 

Von der deutschen Bezeichnung ausgeschlossen sind in diesem Projekt jedoch allgemein anerkannte Java-Konventionen wie \glqq getter\grqq{} und \glqq{} setter\grqq{}. 
IDEs wie Eclipse oder Intelij sind in der Lage diese speziellen Funktionsarten automatisch zu generieren und um diese Funktion zu nutzen, werden diese Funktionsbezeichnungen auch auf Englisch akzeptiert. 
Weiterhin von der deutschen Bezeichnung ausgeschlossen sind Funktionsnamen, Objekte und Variablen, die durch die Verwendung von Processing vorgegeben sind. 
Dazu gehören beispielsweise Funktionen  wie \glqq setup(), settings()\grqq{} oder \glqq draw()\grqq{}.

\section{Domainenbausteine}

Aus der Ubiquitous Language ergeben sich nun die folgenden taktischen Muster des Domain
Driven Designs mit den entsprechenden Objekten.
Dazu wird versucht die Quelldomaine, Schachspiel, so detailgetreu wie möglich in das Programm zu übertragen. 

\newpage

\subsection{Value Objects}

Ein Value Object repräsentiert ein bestimmtes unveränderliches Objekt, das einen bestimmten Wert besitzt und keine eigene Identität aufweist. 
Es definiert sich durch seine Eigenschaften und wird daher als austauschbar und vergleichbar angesehen.

Im Standardschach repräsentiert ein \textbf{Feld} eine Position, die durch eine Kombination aus \textit{Spalte}, \textit{Zeile} und \textit{Farbe} definiert wird. 
Schachfelder haben keinen Lebenszyklus und werden aus diesem Grund als Value Objects betrachtet. 
Es ist von grundlegender Bedeutung für die Funktion eines Schachspiels, dass Schachfelder als unveränderliche Objekte behandelt werden, da ein Schachspiel auf einer \glqq festen Unterlage\grqq{} gespielt wird. 

Ein \textbf{Schachbrett} kann ebenfalls als Value Object betrachtet werden, da es einen unveränderlichen Zustand repräsentiert. 
In der domänenspezifischen Abbildung eines Schachspiels wird es als übergeordnete Verwaltungsstruktur der einzelnen Spielfelder betrachtet. 
Es besteht aus 64 Schachfeldern, die als zweidimensionale Spielebene dargestellt werden.
Die Gleichheit von zwei Schachbrettern basiert auf der Gleichheit ihrer Felder, unabhängig von ihrer tatsächlichen Instanz. 
Da die Felder als Value Objects behandelt werden und ein Schachbrett zu Beginn eines Spiels initial erzeugt und danach unverändert bleibt, wird diese Klasse auch als Value Object geführt. 
Die Integrität des Objekts muss während des Spiels gewahrt bleiben, da es keine Verhaltensänderungen aufweist. 

Ein \textbf{Schachzug} setzt sich aus einer ausgewählten \textit{Figur}, der \textit{Startposition}, der \textit{Endposition} und textuellen Darstellung des Zugs zusammen.
Bei der textuellen Darstellung werden Angaben darüber, ob eine andere Figur geschlagen wird, ob durch den Zug Schach geboten wird, ob eine Bauernumwandlung stattfindet oder ob eine Rochade durchgeführt wird, berücksichtigt. 
Die Implementierung eines Schachzugs als eigene Klasse dient der Protokollierung der gespielten Schachpartie. 
Ein Schachzug wird als Value Objekt klassifiziert, da er keinen erkennbaren Lebenszyklus vorweisen kann. 
Sobald ein Spieler in einem Spiel eine Figur von Position A zu Position B bewegt, wird eine Instanz der Klasse Schachzug erstellt und kann nicht nachträglich bearbeitet werden, da auch im Spiel ein Schachzug nicht zurückgenommen oder geändert werden kann. 

\subsection{Entities}

Eine Entity ist ein Objekt, das innerhalb der Domäne eine einzigartige Identität besitzt und sich im Laufe der Zeit verändern kann. 
Entities werden durch ihre Identität und nicht durch ihre Attribute definiert und können einen Lebenszyklus vorweisen. 

Im Domainencode existiert die Klasse \textbf{Figur}, die als abstrakte Oberklasse für alle Figuren in einem Schachspiel dient. 
Da die Figur als \textit{abstract class} implementiert ist, kann sie nicht instanziiert werden. 
Aus diesem Grund wird nicht die Klasse Figur selbst, sondern alle Spielfiguren, die von Figur erben, als Entitäten zu betrachten. 
Zu diesen Spielfiguren gehören \textbf{König, Dame, Turm, Läufer, Springer} und \textbf{Bauer}.
Jede Figur kann über das Attribut \textit{position} eindeutig identifiziert werden.
Der Schlüssel, den das genannte Attribut darstellt, ist kein natürlicher Schlüssel, da es kein wirklicher Identifikator aus der Domaine ist, sondern eine für den Betrachtungszeitpunkt eindeutige Eigenschaft der Figur. 
Da auf einem Feld im Schach immer nur eine Figur stehen kann, ist die Eindeutigkeit durch die Positionen gewährleistet. 
Wenn eine Figur auf ein Feld zieht, auf dem sich bereits eine andere Figur befindet, wird die Figur, welche ursprünglich auf dem Feld stand, geschlagen und aus dem Spiel entfernt. 
Somit steht letztendlich nur eine Figur auf dem Feld. 
Innerhalb der Domäne \glqq Schach\grqq{} ist dieser Eigenschaftsschlüssel mit jedem Schachzug veränderlich, da sich die Schachfiguren auf dem Spielfeld bewegen können. 
Die Verwendung eines Surrogatschlüssels wäre grundsätzlich auch möglich, aber in diesem Fall wird sich so gut es geht an der Domaine orientiert. 
Im analogen Schach wird der Figurentyp und die Position zur eindeutigen Identifizierung verwendet. 
Die Farbe kann indirekt durch die Anordnung der Angaben bestimmt werden.
Jede Schachfigur besitzt einen eigenen Lebenszyklus.
Sobald ein Schachspiel startet, werden alle Figuren in ihrer Startposition instanziiert.
Während des Spiels können die Figuren nahezu unbegrenzt bewegt werden. 
Wenn eine Figur geschlagen wird, wird sie wieder gelöscht. 

Obwohl ein Schachzug ein Value Object ist, wird in diesem Projekt der \textbf{Spielzug} als Entity gewertet, obwohl nur zwei Schachzüge mit einer ID zusammengefasst werden. 
Der Spielzug lässt sich eindeutig durch eine \textit{zugNummer} identifizieren, die mit jedem Spielzug um eins inkrementiert wird.
Da diese Art der Zählung und Zugidentifizierung ebenfalls in der Domaine verwendet wird, ist die Einordnung als natürlicher Schlüssel passender als eine Bezeichnung als Surrogatschlüssel.
Der Grund, warum der Spielzug als Entity und nicht als Value Object gehandelt wird, liegt am Lebenszyklus des Objekts. 
Nach dem Erstellen einer Instanz des Spielzugs wird zuerst der Zug des Spielers \glqq Weiß\grqq{} gesetzt.
Im zweiten Schritt wird der schwarze Zug hinzugefügt, bevor der Schachzug zur Protokollierung der Schachpartie verarbeitet und final wieder verworfen wird.  

Die Klasse \textbf{Schachspiel} sollte als Entity behandelt werden, da sie eine eindeutige Identität besitzt und sich im Verlauf der Schachpartie in verschiedensten Zuständen befindet. 
Sie bildet die übergeordnete logische Ebene, welche eine bestimmte Figurenkonstellation auf dem Schachbrett zu einem bestimmten Zeitpunkt zusammenführt. 
Im Schach sind pro Zug durchschnittlich 35 unterschiedliche Züge möglich, was die Anzahl der Stellungen in einem Schachspiel exponentiell steigen lässt, abhängig von der Anzahl bereits gespielter Züge, der Positionierung der Figuren und der Anzahl der geschlagenen Figuren. 
Obwohl die Ausgangslage der Figuren immer gleich ist, unterscheidet sich die Partie Schach individuell sehr stark in ihrem Verlauf.
Um die Schachspiele eindeutig zu identifizieren, bietet sich die Verwendung eines Surrogatschlüssels an.
In der verwendeten Programmiersprache Processing, einer Java-Erweiterung, eignet sich hierfür die Verwendung einer \textit{UUID}.

\subsection{Aggregate}

Aggregate ermöglichen die Verwaltung von Entities und Value Objects als eine zusammengehörende Einheit.
Ein Aggregat besteht dabei aus mindestens einer Entität und kann weitere Entitäten und Value Objects enthalten.

\begin{figure}[ht!]
    \centering
    \includegraphics*[scale=0.6]{Bilder/DDD_Aggregate_v2.PNG}
    \caption{Ausschnitt aus dem Domainenmodell eines Schachspiels}
\end{figure}

In der Domaine \glqq Schach\grqq{} können die Klassen \textbf{Schachbrett} und \textbf{Feld} als Aggregat zusammengefasst werden, da sie logisch miteinander verknüpft sind. 
Für ein Schachbrett werden 64 Instanzen der Klasse Feld in einer zweidimensionalen Datenstruktur gespeichert. 
Da über das Schachbrett in der Domaine die gesamte Handlung des Spiels stattfindet (Figuren werden über das Schachbrett bewegt), hat die Klasse Schachbrett eine zentrale Rolle.
Um die feste Anordnung der Felder im zweidimensionalen Rahmen des Spielfeldes durch Zeilen und Spaten positionsgetreu abzufragen, kann auf jedes Feld auf dem Schachbrett nur über das übergeordnete Schachbrett zugegriffen werden. 
Durch diese Art des Zugriffs fungiert die Klasse Schachbrett automatisch als \textit{Root-Entity} für die Klasse Feld.
Zwischen der Root-Entität Schachbrett und der Entität Schachspiel kann die Assoziation nicht durch eine Entity-ID gelöst werden, da es sich bei dem Schachbrett um ein Value Object handelt und einem Schachbrett daher keine ID zugewiesen wird. 

Die Klasse \textbf{Spielzug} dient ebenfalls als \textit{Root-Entity} für das Aggregat, das aus den Klassen  Spielzug und \textbf{Schachzug} besteht. 
In der Domaine kann jeder Spieler, wenn er an der Reihe ist, einen Schachzug ausführen, indem er eine Figur von Position A nach Position B bewegt. 
Ein Spielzug fasst jeweils zwei aufeinanderfolgende Schachzüge von Spieler \glqq Weiß\grqq{} und Spieler \glqq Schwarz\grqq{} zusammen. 
Die schriftliche Protokollierung einer Partie Schach erfolgt immer über die Spielzüge und gibt jeweils die einzelnen Schachzüge der unterschiedlichen Spieler an. 
Auf diese Weise kann man auf die einzelnen Schachzüge nur über die Spielzüge zugreifen, was für eine Betrachtung als Aggregat spricht.  

\subsection{Repositories}

Im Domain Driven Design werden Repositories genutzt, um Daten in einem Projekt zu persistieren und zu laden. 

Im Falle des programmierten Schachspiels soll nach jedem abgeschlossenen Spiel ein lesbares Protokoll in eine eigene Datei exportiert werden.
Obwohl die Daten nie gelesen werden, ist es erforderlich, dass für die Klasse Spielzug ein Repository verwendet wird, um die Spielzüge nacheinander in der Protokolldatei zu persistieren. 

In den Vorlesungsfolien wird beschrieben, dass meist zu jedem Aggregat auch ein entsprechendes Repository existiert. 
Für das Aggregat Schachbrett ist dies nicht der Fall. 
Das Schachbrett muss in keinem Fall persistiert werden und darum wird kein Repository benötigt. 

\subsection{Domain Service}

Ein Domain Service ist ein Konzept, das genutzt wird, um Logik abzubilden, die nicht direkt mit einer Entität oder einem Aggregat verbunden ist.

Im Schachspiel sind alle Figuren als Entitäten zu betrachten, wobei jede Figur eigene Regeln hat, nach denen sie sich auf dem Schachbrett bewegen darf.
Die Funktionalität, die die möglichen Bewegungen der verschiedenen Figurtypen ausgehend von ihrer aktuellen Position wiedergibt, befindet sich in den Klassen \textbf{Bewegungsmatrizen} und \textbf{Bewegungsrichtung}.
Beide Klassen können daher als Services in der Domaine betrachtet werden.
